Someone wanted me to write out my thoughts as coherently as possible as
to where the streaming television station project stood so that blanks
could be filled in. Somebody has to clear up the gaps. This one is not a
solo project in the long term. What follows below is a bit of a brain
dump.

The streaming television station project serves a need as the local news
sources are not the most functional. Although
\href{https://vimeo.com/user126359532}{Conneaut CAT TV} does exist it is
limited in its scope and its quality is lacking. The quality issue was
raised after discussion with a member of the staff of a television
station in Cleveland as to our local attempt at a video newscast.
Currently service from our local radio stations in Ashtabula County does
not have consistent coverage of local news. Newspaper reporting locally
is a shadow of what it was pre-pandemic.

The goal initially would be to start with offering a ``headlines
service'' which would recap the headlines in a manner akin to top of the
hour news bulletins from BBC World Service/CBC Radio One. Growth would
be on an iterative basis to move from a headlines service toward a more
traditional American evening news broadcast. Initial lengths of news
broadcasts would be targeted at five minutes and then would increase
over time to thirty minutes. The tentative list of topical coverage
areas includes:

\begin{itemize}
\tightlist
\item
  Ashtabula City Council \& other municipal affairs
\item
  Ashtabula Area City Schools
\item
  Ship traffic in the Ashtabula Harbor
\item
  Community events with the churches
\item
  Community events with the social organizations
\item
  Crime
\item
  Finance
\item
  Medical news with our local hospital, Ashtabula County Medical Center
\item
  Parks events
\item
  Business news, including chamber of commerce updates
\end{itemize}

There needs to be at least two reporters to cover events in the
identified topical areas. Ideally they would be
\href{https://en.wikipedia.org/w/index.php?title=Multimedia_journalism&oldid=1131239568}{multimedia
journalists} to be able to handle videography on their own in the old
style of CityTV across the lake in Ontario. I'm thinking specifically of
the
\href{https://en.wikipedia.org/w/index.php?title=CP24&oldid=1138757831}{CP24}
news style for presentation from the days when
\href{https://en.wikipedia.org/w/index.php?title=Mark_Dailey&oldid=1124036072}{Mark
Dailey} was still a presenter. If we have to have a cameraman in reserve
then we would try to make that a cross-training responsibility for
somebody with other duties. There would still need to be a host for
continuity purposes but that role could be alternated between the
multimedia journalists.

On the back-end there have to be some production staff. There have to be
technical staff
\href{https://en.wikipedia.org/w/index.php?title=Production_control_room&oldid=1087451141}{to
do editing and mix down to produce the show} as well as
\href{https://en.wikipedia.org/w/index.php?title=Broadcast_designer&oldid=1096142751}{make
it look good}. There have to be people to
\href{https://en.wikipedia.org/w/index.php?title=Transmission_control_room&oldid=1107297485}{handle
the computer connections to get the broadcast actually going out to the
world}. That's keeping
\href{https://en.wikipedia.org/w/index.php?title=Television_crew&oldid=1121314540}{operational/creative
crew} separate from
\href{https://en.wikipedia.org/w/index.php?title=Television_producer&oldid=1133167514}{administrative/financial}.
Yeah, a producer would be needed to handle the administrative side so
the production side could stay functioning.

A small team of about five to seven would be needed to staff the project
and spin it up. Going solo results in things like
\href{https://vimeo.com/803474271}{this from Conneaut CAT TV} and that's
an unacceptable outcome. Even with a small team of five to seven the
operation wuold be running very lean with no real margin for error.

Missing from this is an advertising operation. Using monetization
options from the online video hosting platform would be likely the best
thing to do at first. The overall ad market nationally is unusually soft
and that has been reflected in cuts and closures at both radio and
television stations. Even if a miracle worker was hired for ad sales
they would be placed in a no-win situation based upon the state of the
market.

Hosting would be managed via Vimeo. To handle live streaming via Vimeo
would cost \href{https://vimeo.com/upgrade}{\$75 per month} at this
time. Sticking to a broadcast paradigm appears to be the best way to
meet the needs of local viewers rather than utilizing a video-on-demand
model. A VOD model would likely be implemented via video podcasting but
that would be harder to monetize and more difficult for local viewers to
access. Vimeo does have the capability to restrict broadcasts to paid
subscribers only although that would cost more to do.

Appropriate production space would be needed. Essentially a set would
need to be constructed. Although tools like
\href{https://obsproject.com/kb/chroma-key-filter}{OBS Studio allow for
a decent amount of chroma key work} the problem comes down to having
sufficient room in one's home to be able to record. Securing an actual
place to record in would obviate many of the problems on that front.
Spaces in which this could be done are priced at prohibitive levels
currently. There are articles about virtual production techniques but
the one at
\url{https://www.tvtechnology.com/news/the-reality-of-virtual-production}
deals more with using LED walls as well as the notion of ``broadcast
virtual sets'' but they still require a degree of \emph{space} to work
in. It appears that \href{https://www.newtek.com}{Newtek} is pretty much
the main vendor for the technology to back up such things since the
outbreak of the COVID-19 pandemic.

The ``minimum viable product'' in this instance would be to produce a
five minute evening news broadcast every day that could be viewed by
local viewers. In this minimum viable product we would also cover local
stories across our communities. We would read out the weather reports
from the National Weather Service forecast office in Cleveland to begin
with although that would expand hopefully in the future. The goal would
be to do something akin to a BBC World Service/CBC Radio One top of hour
news break. I think we can manage to accomplish that.

That's what I've got. Prevailing wage data would be a concern since
Ashtabula County would rightfully be considered a ``hardship posting''.
Searches of the data at \url{https://flcdatacenter.com/} and considering
numbers at 110-120\% of the rates shown would give good figures to start
with for budgeting personnel expenses. Since people would have to be
hired from outside the county to staff this there would have to be a
premium paid to secure their services.

There's currently no piggy bank available to raid for something like
this. A crowdfunding effort using Indiegogo or Kickstarter would most
likely be needed. Doubts exist as to the amount of local participation
that might occur with such a fundraising appeal.

The foregoing is the most coherent brain dump I can come up with at this
point.
