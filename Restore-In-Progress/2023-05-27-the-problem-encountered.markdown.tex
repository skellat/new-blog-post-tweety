I don't normally listen to the \href{https://2.5admins.com/}{2.5 Admins
podcast}. When I saw that the
\href{https://2.5admins.com/2-5-admins-144}{latest episode} was talking
about AM Radio I figured I would tune in. The trials of AM Radio have
been a bit of a topic on the \href{http://www.worldofradio.com/}{World
of Radio} discussion list that I participate in after all.

The problem encountered with the episode was how confused the discussion
got around AM radio. I wanted to break down a few matters in a blog post
since they were not handled all that well by the podcast. Most of it
revolves around emergency notifications to the public.

In the United States we have AM radio playnig a key role in the
Integrated Public Alert \& Warning System (IPAWS). The Emergency Alert
System (EAS) is just \emph{one single component} to IPAWS. The Wireless
Emergency Alert (WEA) is another component to it. NOAA Weather Radio
(NWR) also falls under IPAWS when it comes to providing alert
information. The Common Alerting Protocol that runs IPAWS is XML-based
so conceivably it can be extended in the future to do things nobody has
thought of yet. Having IPAWS provide a one-way pipeline to the various
generative AI chatbots could get interesting provided those bots could
reliably convey the information from the feed.

There are 70 or so ``primary entry points'' nationally for messages to
originate in the system. They're all reinforced AM radio stations that
happen to be on historic ``clear channel'' spots on the broadcast dial
with large broadcast contour footprints. The closest national primary
entry point to me is WTAM AM in Cleveland, for example. The federal
government has paid to upgrade and reinforce those radio stations to
keep them running in the event bad things happen.

As to the alerting system, it should be looked at as being similar to
Markdown. A human being can look at something raw that was styled in
Markdown and generally still get the meaning without needing a parser.
The use of a parser allows for greater access to richer content.

The same thing goes for alert message blasts on the radio dial. By
itself a human being will hear the shrill noises at the start of the
message to warn them something is wrong and they'll wait for the voice
message to play out. Those shrill sounds at the beginning of an alerting
message are actually data that you can hear. The data is being pushed in
the style of audio frequency shift keying at an odd bit rate with even
odder mark and space tones for the text blast. Suitable receivers can
decode those blasts at the start of messages and pop up categorical
warnings for listeners or even tell listeners if a message is \emph{not}
for their area.

AM is relatively simple to receive. The episode cited crystal radios as
being no-power possibilities. Considering the level of disaster this
country has seen ranging from forest fires destroying much of the west
to hurricanes wiping out the Caribbean, we want to reach people who may
have had their means of tuning in frankly destroyed. Tuning in FM with a
crystal radio is a matter we need not discuss here.

Yes, electric vehicles are very noisy electrically and put out quite a
bit of radio frequency interference. They may not pollute with
hydrocarbons but they pollute in their own way. I haven't looked to see
if electric vehicles are banned out in the national radio quiet zone in
West Virginia or not but I assume they are.

AM radio has its place. It has been a lifeline keeping the community
informed in Puerto Rico when disaster strikes. It serves a key role in
keeping IPAWS running. It serves you even when cell phones encounter
problems trying to disseminate emergency information as
\href{https://www.fcc.gov/document/wireless-emergency-alerts-2022-performance-exercise-report}{found
in an FCC tech report on WEA operations issues}.

Overall the episode was good. The AM radio topic just continues to get
weird for those of us with operational knowledge of the field. Hopefully
things improve\ldots{}
