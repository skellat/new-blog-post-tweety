The \href{https://xubuntu.org/news/xubuntu-23-04-released/}{ISOs for
Ubuntu 23.04 dropped on Thursday}. I downloaded a server ISO for arm64
so I could fiddle with it using UTM/QEMU on my MacBook Pro M1. It worked
in terms of getting it fired up. The problem is that even though I had
``virtualization'' selected the laptop managed to get very, very warm as
I was doing the install. The pure Debian VM doesn't do that!
Virtualization and emulation can pound the crap out of a system if you
don't have your settings configured properly so I may have some tuning
yet to do.

I did get
\href{https://en.wikipedia.org/w/index.php?title=Fldigi&oldid=1148184949}{fldigi}
installed on both the MacBook and the clunker Windows 10 laptop that I
use when on-site with the employer. This is a very useful tool. Using it
with an appropriate radio would allow for decoding
\href{https://www.weather.gov/marine/uscg_broadcasts}{marine weather
broadcasts} on a variety of channels. The text broadcasts and image
broadcasts would be useful for some of the backburner projects like the
\emph{samizdat} newspaper effort. Broadcasts specific to our region in
the Lower Great Lakes are
\href{https://www.canada.ca/en/environment-climate-change/services/general-marine-weather-information/understanding-forecasts/regional/products-services-great-lakes.html}{produced
by Canadian authorities} and
\href{https://www.ccg-gcc.gc.ca/publications/mcts-sctm/ramn-arnm/part4-eng.html}{picking
up the NAVTEX broadcast on 518 kHz} would help nicely in setting up such
proposed works.

There are things to be done, I guess. The debt ceiling crisis is still
unresolved. Planning ahead in these strange weather conditions is not
the best idea.

Oh well, things on the tech side of the house seem to be the most
positive lately for me\ldots{}
