I have done some tweaking to the blog to try to get it working in a more
conventional fashion. Old posts will be found on the archive page. The
front page will show only \emph{five} posts. The RSS feed will show only
\emph{twenty} posts. Everything will still live in the git repository
that this is built from.

The vodcasting experimentation site at \url{https://coyote.works/} has
had similar changes made to it. If I cannot get the streaming situation
figured out within the next two weeks we'll simply shift to launching
vodcasting as a public service in our local commnity. Once the viewers
have their podcatching set up appropriately they would be in good shape.
The big question is how much effort would viewers be willing to expend
since it would \textbf{not} be found within Facebook.

Considering the intermittent power outages that happened throughout the
day on New Year's Eve I will need to look at commercial real estate. The
closest data center is apparently in Richmond Heights and is owned by
\href{http://www.ncsdata.com/datacenter.html}{NCS Datacom}. As much as I
would like to do something with
\href{https://yunohost.org/en/whatsyunohost?q=\%2Fwhatsyunohost}{YunoHost}
at home it isn't safe to do that in terms of infrastructure.

It is not like
\href{https://commercial.century21.com/real-estate/ashtabula-oh/LCOHASHTABULA/?kw=&pt=1,5,4,6,8}{commercial
real estate} is bountiful locally in decent locations. My little century
home doesn't exactly have space to work out of. This is a bottleneck to
have to work through.

Many challenges lay in wait in the year ahead, it seems.
