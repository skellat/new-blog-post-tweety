Yes, it is time to talk code snippets again. In this instance we're
talking the invocation of various commands in bash/zsh. We look first at
\href{https://github.com/yt-dlp/yt-dlp}{yt-dlp}. Consider:

\begin{Shaded}
\begin{Highlighting}[]
\ExtensionTok{yt{-}dlp} \AttributeTok{{-}{-}xattr{-}set{-}filesize} \AttributeTok{{-}{-}restrict{-}filenames} \AttributeTok{{-}{-}continue} \AttributeTok{{-}{-}embed{-}thumbnail} \AttributeTok{{-}{-}video{-}multistreams} \AttributeTok{{-}{-}audio{-}multistreams} \AttributeTok{{-}{-}write{-}auto{-}subs} \AttributeTok{{-}{-}embed{-}subs} \AttributeTok{{-}{-}embed{-}metadata} \AttributeTok{{-}{-}embed{-}chapters} \AttributeTok{{-}{-}xattrs} \AttributeTok{{-}{-}concurrent{-}fragments}\NormalTok{ 32 }\AttributeTok{{-}{-}downloader}\NormalTok{ aria2c }\AttributeTok{{-}{-}downloader{-}args} \StringTok{"{-}s16 {-}j16 {-}x16"} \AttributeTok{{-}{-}windows{-}filenames} \AttributeTok{{-}{-}trim{-}filenames}\NormalTok{ 64 }\AttributeTok{{-}{-}mtime} \AttributeTok{{-}{-}no{-}remove{-}chapters} 
\end{Highlighting}
\end{Shaded}

Working backwards in terms of the selected options, this invocation of
yt-dlp will:

\begin{itemize}
\tightlist
\item
  Do not remove any chapters from the file (default)
\item
  Use the Last-modified header to set the file modification time
  (default)
\item
  Limit the filename length (excluding extension) to the specified
  number of characters (64)
\item
  Force filenames to be Windows-compatible
\item
  Use \emph{aria2c} as an external downloader and pass arguments to that
  external program
\item
  Number of fragments of a dash/hlsnative video that should be
  downloaded concurrently (default is 1) (Chosen is 32)
\item
  Write metadata to the video file's xattrs (using dublin core and xdg
  standards)
\item
  Add chapter markers to the video file (Alias: --add-chapters)
\item
  Embed metadata to the video file. Also embeds chapters/infojson if
  present unless --no-embed-chapters/--no-embed-info-json are used
  (Alias: --add-metadata)
\item
  Embed subtitles in the video (only for mp4, webm and mkv videos)
\item
  Write automatically generated subtitle file (Alias:
  --write-automatic-subs)
\item
  Allow multiple audio streams to be merged into a single file
\item
  Allow multiple video streams to be merged into a single file
\item
  Resume partially downloaded files/fragments (default)
\item
  Restrict filenames to only ASCII characters, and avoid ``\&'' and
  spaces in filenames
\item
  Set file xattribute ytdl.filesize with expected file size
\end{itemize}

Some of those may be duplicative and overlapping in scope. Some of those
are defaults and don't need to be specified. This is the base set of
options that I use to archive videos by my church from YouTube. In
theory the settings can be rearranged to only grab the subtitles file so
that that can be mangled into being a transcript.

The yt-dlp tool is very useful at this point in life. Downloading
something completely \emph{first} prior to playback helps ensure smooth
viewing experiences. Our local broadband can be sketchy at the most
inopportune times.

Now we shift to considering
\href{https://wkhtmltopdf.org/}{wkhtmltopdf}. Our stanza in question
looks like:

\begin{Shaded}
\begin{Highlighting}[]
\ExtensionTok{wkhtmltopdf} \AttributeTok{{-}{-}image{-}quality}\NormalTok{ 100 }\AttributeTok{{-}{-}image{-}dpi}\NormalTok{ 600 }\AttributeTok{{-}{-}margin{-}bottom}\NormalTok{ 2.54cm }\AttributeTok{{-}{-}margin{-}left}\NormalTok{ 2.54cm }\AttributeTok{{-}{-}margin{-}right}\NormalTok{ 2.54cm }\AttributeTok{{-}{-}margin{-}top}\NormalTok{ 2.54cm }\AttributeTok{{-}{-}orientation}\NormalTok{ portrait }\AttributeTok{{-}{-}page{-}size}\NormalTok{ letter }\AttributeTok{{-}{-}use{-}xserver} \AttributeTok{{-}{-}background} \AttributeTok{{-}{-}default{-}header} \AttributeTok{{-}{-}enable{-}external{-}links} \AttributeTok{{-}{-}enable{-}forms} \AttributeTok{{-}{-}images} \AttributeTok{{-}{-}enable{-}internal{-}links} \AttributeTok{{-}{-}enable{-}javascript} \AttributeTok{{-}{-}load{-}error{-}handling}\NormalTok{ skip }\AttributeTok{{-}{-}load{-}media{-}error{-}handling}\NormalTok{ skip }
\end{Highlighting}
\end{Shaded}

This invocation effectively turns off compression of images, specifies
an image dpi of 600, puts one inch margins on the generated PDF pages,
sets ``portrait'' orientation, sets letter paper size, and slaps a
default header onto what is grabbed. Backgrounds are rendered as are
external links, forms, images, and internal links. Error handling is
handled by just skipping things that refuse to load.

This tool is great for taking a nice snapshot of a cooperating web page.
It isn't perfect, though, as it will barf on things that have tracking
pixels. Trying to run it on a page from the site of my local daily
broadsheet is almost impossible due to the number of errors that erupt.

There are other tools out there for archiving but these are low-level
ones that are useful.
