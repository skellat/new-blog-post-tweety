Terence Eden put up a blog post
\href{https://web.archive.org/web/20230421163049/https://shkspr.mobi/blog/2023/04/how-do-you-decentralise-emergency-alerts/}{talking
about how to replace the emergency alerts function} that Twitter
previously served and now has effectively broken. The post begins to put
forward ideas on how to reinvent what existed on Twitter. My problem is
that it overlooks that much of this is already worked through. Rather
than build completely new software to handle the matter from scratch we
could simply hit the appropriate APIs sooner and more appropriately.

Things like the
\href{https://www.fema.gov/emergency-managers/practitioners/integrated-public-alert-warning-system}{Integrated
Public Alert and Warning System} already exists in the United States of
America. From there we have the
\href{https://www.fema.gov/emergency-managers/practitioners/integrated-public-alert-warning-system/technology-developers/ipaws-open}{Open
Platform for Emergency Networks} or IPAWS-OPEN. I mean, even PBS builds
off that API to have \href{https://warn.pbs.org}{a warnings map to see
what they look like across the entirety of the country} and the IPAWS
program office
\href{https://www.fema.gov/event/pbs-warn-your-live-wea-map}{had a
webinar about this} recently. The
\href{https://www.fema.gov/emergency-managers/practitioners/integrated-public-alert-warning-system/technology-developers/common-alerting-protocol}{Common
Alerting Protocol} is the API in question.

Now, I'm not an Authority Having Jurisdiction in this case. I can't make
decisions about the EAS system as I'm not in charge of it. All I can
tell folks is to monitor more than one system for updates in severe
weather. That means having a NOAA Weather Radio receiver is a very, very
prudent investment presently.

Could a botsinspace account be created to restore these auto-posts of
National Weather Service reviews? As far as I can tell, the answer to
that is yes. Is it likely? No, not at all.

IPAWS is already configured to use so many communications routes as it
is. Will an ActivityPub-based repeater add much value at this point? I'm
just not sold on it.
