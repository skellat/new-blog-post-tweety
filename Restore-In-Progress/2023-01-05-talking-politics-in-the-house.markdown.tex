\begin{quote}
\emph{The House of Representatives shall chuse their Speaker and other
Officers; and shall have the sole Power of Impeachment.}

---Clause 5, Section 2, Article I, United States Constitution
\end{quote}

For the purpose of clarifying things for European and British colleagues
I will try to go over a few things relative to proceedings today in the
US House of Representatives. We are now at eleven failed votes to elect
a Speaker. There is no real indication that the stalemate is going to
break any time soon.

The House is now past the point its predecessor House in 1923 was when
it
\href{https://history.house.gov/People/Office/Speakers-Multiple-Ballots/}{took
nine votes to elect a Speaker}. We are now in the sort of territory last
seen by the House right before the Civil War broke out. At that time it
\href{https://history.house.gov/People/Office/Speakers-Multiple-Ballots/}{took
hundreds of ballots over months for a Speaker to be finally elected}.

Other than the second section of the first article of the federal
constitution and the twenty-fifth amendment to the constitution there is
no other mention in it of the Speaker. There is
\href{https://web.archive.org/web/20210305112024/https://fas.org/sgp/crs/misc/97-780.pdf}{no
mention of qualifications}. All the constitutional text says is that the
House shall choose its Speaker.

Based upon the established procedures of the Congress, none of the
members can be sworn in until the Speaker is elected. Procedures can not
be changed until they are sworn in either. Once they are sworn in they
can change the procedures for \emph{next time}. Whether we like it or
not this challenge is what we have to get past before things can
proceed. Having such a narrow majority is a choice the American people
made and they have to live with it.

This comes down to an absolute majority of all members-elect voting. No
plurality wins are possible otherwise Hakeem Jeffries would have won any
of the last eleven votes. The ways out are pretty limited. Any
concessions that McCarthy makes to win over the rebels still have to be
approved by the rest of the GOP caucus and if they are too extreme the
caucus might turn on him. Kevin McCarthy has not won over the rebels
with concessions so far and has not much more he can concede other than
dropping out. He can ensure they miss the vote somehow and not cast
votes but that would result in Hakeem Jeffries winning. Five members of
the Republican caucus could vote for Jeffries and not just stop the
madness but also their congressional careers. The rebels could find a
candidate the entire caucus could get behind and kick McCarthy to the
curb.

None of those ways out are all that likely to be pursued at this point.
The farce will continue at least through Friday. When we get to the
fourteenth vote, will anybody have a change of heart then? I am
doubtful.

There is a practical deadline to this. As someone who had to suffer
through \href{https://www.gao.gov/products/gao-20-377}{the thirty five
day government shutdown} I know the deadline well. A Speaker must be in
place and the House must be functional before the fiscal year ends on
September 30th at 11:59 PM. The Senate cannot act on its own
\href{https://www.senate.gov/about/powers-procedures.htm}{except in some
cases not directly related to legislating}.

Of course, now that we know we have a team of twenty bent of obstruction
the odds are pretty good that we will see a government shutdown in
October. Even if a Speaker gets elected the odds of any legislation
getting passed are extremely slim. I am very doubtful we will even see a
continuing resolution let alone appropriations bills by the end of the
fiscal year.

These will be way more interesting times than I expected this year.
