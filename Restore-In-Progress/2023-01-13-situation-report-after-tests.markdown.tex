Sometimes things \emph{cannot} be crammed into a single fediverse post.
A proper blog post has to be made to express all that is happening. In
this instance tests were conducted and were not successful. There will
need to be some realignment of operating plans, I think.

After the recent lack of success with chroma key filter usage we did
some re-work on the DIY green screen. We also did some realignment of
lighting within available resources. We found that we \emph{could} get
the green screen mostly lit but wound up with an overly hot spot near
the bottom right part of the screen plus one seam of the screen didn't
hold together correctly. Even then we could not get the person in front
of the screen properly lit.

My little ``office'' space at home was originally a small child's room
when the house was built over one hundred years ago so there is only
electrical panel in the fairly small room. There is not enough
electrical power or space to bring in more lighting to that room. Even
though we \textbf{do} have the technological capability to do something
like this we are hampered by the more theatrical aspects.

The project is not lost, though. We can reduce scope. The
\href{https://code.launchpad.net/~skellat/+git/show-prep-scripts}{scripting
on Launchpad} is essentially neutral in its possible use. It could just
as easily be applied to posting a narrated slideshow. After all, that
was part of the shtick of
\href{https://en.wikipedia.org/w/index.php?title=The_Weather_Channel&oldid=1130419446}{The
Weather Channel} for years through use of the
\href{https://en.wikipedia.org/w/index.php?title=Weatherscan&oldid=1129582881}{Weatherscan}
system.

Yes, this is a setback. Sometimes to get ahead we have to take a step
back. Before we can even approach a fundraiser I think we have to do
narrated slideshows for a while. I suspect I could build up the
automation of production as time goes by, too.

We'll see what erupts, eh?
