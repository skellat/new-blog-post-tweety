The law of unintended consequences has not been repealed, alas. State
Senator Jerry Cirino proposed a bill called the
\href{https://www.legislature.ohio.gov/legislation/135/sb83}{``Ohio
Higher Education Enhancement Act''}. My local state senator, the
infamous Sandy O'Brien, is a co-sponsor. In this bill it is proposed to
cut off state financial assistance to private colleges in Ohio unless
they affirm they hire faculty without any ``ideological litmus test''.
When you read further through the bill you find that that is \emph{not}
adequately defined. The concept arises in the proposed section
3345.0217(B)(9) which states:

\begin{quote}
\emph{(9) Prohibit political and ideological litmus tests in all hiring,
promotion, and admissions decisions, including diversity statements and
any other requirement that applicants describe their commitment to a
specified concept, specified ideology, or any other ideology, principle,
concept, or formulation that requires commitment to any controversial
belief or policy;}
\end{quote}

Why is this anti-Christian? Consider the map of
\href{https://highered.ohio.gov/about/ohios-campuses/map}{private
colleges and universities authorized to operate in Ohio} posted by the
Ohio Department of Higher Education. They're listed as ``Independent
Colleges and Universities''. If you look at the list you'll see a bunch
that offer education in religion and ministry such as
\href{https://www.franciscan.edu/}{Franciscan University of
Steubenville}, \href{https://www.ashland.edu/}{Ashland University},
\href{https://www.ohiochristian.edu/}{Ohio Christian University}, and
others. There is an
\href{https://www.eeoc.gov/employers/small-business/hiring-decisions-based-religion}{exception
in federal law} that allows such institutions to employ people only of
their faith tradition to train their own ministers to go out into the
world. This bill from State Senator Cirino would cut those institutions
off from Ohio-provided institutional funding supports if they take
advantage of that provision of federal law to ensure that they maintain
their own minister training programs rather than having to hire people
of other faith traditions or no faith at all. The schools would be left
with the horrible choice of losing state funding supports or jettisoning
ministry training programs.

Overall this is another ``divisive concepts'' bill. Supposedly these
state legislators mean well. I don't think they actually care about the
consequences to these things.
