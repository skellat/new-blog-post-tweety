One of the ways to get past the lack of room is to start implementing
groupware. I recently had the chance to consider unique ways Microsoft
Teams could be used to provide for a ``virtual office''. At present the
commercial real estate market in the Ashtabula area is just not that
good. There are plenty of unusable properties that have overinflated
prices attached to them and frankly nothing that is anywhere close to
usable.

How would Teams provide for a ``virtual office''? Assuming I could find
collaborators I would end up having to provision the Cloud PBX and other
telephony options. We would need ways to receive phone calls as well as
be able to call each other directly. Teams also interoperates with OBS
Studio so a person could use it to stream their video from another
location to my own for recording. A theoretical use in that case would
be to have two presenters at different locations piping their
audiovisual content over Teams back to me to route and mix down. Storage
via OneDrive for Business would allow for easier transfer of bulky rich
media files.

Could this be done with a hosted NextCloud service vendor? The answer to
that is \emph{maybe}. It wouldn't be nearly as slick as the unified
communications component is not as tightly integrated. I would likely
need to set up Asterisk, look at the starfish book, and contract with
somewhere like \texttt{voip.ms} for a SIP trunk line to the Public
Switched Telephone Network. None of this includes the video conferencing
component that would still need to be set up. OBS Studio has
integrations out of the box to work with
\href{https://techcommunity.microsoft.com/t5/modern-work-app-consult-blog/how-to-host-a-live-streaming-with-multiple-participants-using/ba-p/1291745}{Skype},
\href{https://adoption.microsoft.com/en-us/inside-microsoft-teams/bonus-clips/connecting-microsoft-teams-obs/}{Teams},
and Zoom. Anything else would require looking for potentially an add-on
to OBS Studio or different recording orchestration software due to
\href{https://obsproject.com/kb/virtual-camera-guide}{limits}.

Do I like going down this path? No, not really. I'm just gobsmacked at
people locally asking for a quarter of a million dollars for tiny
commercial buildings that are no bigger than three portable outhouses
placed side by side. I could understand pricing like that in New York
City but not in Ashtabula City. It is almost a form of ``malicious
compliance'' by putting something up for sale to meet someody's
requirements but setting the price absurdly high so that no buyer would
touch it.

There is plenty more to think about in this case, I think.
