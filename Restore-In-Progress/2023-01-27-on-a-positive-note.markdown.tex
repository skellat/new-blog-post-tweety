Please forgive this being somewhat stream of consciousness in its style.
Unfortunately this stream is meandering and doubling back on itself at
points.

I am working on ideas as to how to surmount the problem of having no
production space in which to produce ``television'' locally. The key at
this point is going to likely be videoconferencing. There are many
things you can do on that front. I'm going to have to mine texts like
Chris Kenworthy's
\href{https://www.librarything.com/work/626712/book/187591453}{Digital
Video Production Cookbook: 100 Professional Techniques for Independent
and Amateur Filmmakers} and Lenny Lipton's
\href{https://www.librarything.com/work/531879/book/212302420}{Independent
film making} for production ideas that can be adapted. Video
conferencing and green screens can provide a virtual stage to work from
for the time being.

The format and style for any news broadcast is probably going to emulate
the \href{https://archive.org/details/randomaccess_2}{Random Access}
segment used by
\href{https://archive.org/details/computerchronicles}{Computer
Chronicles}. That's a far better format than the
\href{https://vimeo.com/792662413}{Hometown Happenings program on
Conneaut CAT TV}. At best we can do a quick headlines service in the
style of CNN Headline News from the late 1990s. That would be the best
way to leverage what resources we \emph{do} have.

Quick headlines daily at a time I have yet to figure out in video
podcast format is the target. With video conferencing, judicious use of
already available tech tools, prudent purchasing of limited tech
supplies, and lots of patience I do think \emph{something} could be
bootstrapped. The initial planned scope was always meant to restrict
such a service to start with just Ashtabula City plus the townships of
Ashtabula, Saybrook, Plymouth, and Austinburg. That's a pretty big area
by itself in the state's largest county by area.

Collaborators would be needed to cover government meetings, school
events, and monitor scanner traffic to see what can be heard on the
MARCS channels. Yeah, that sounds almost like a normal newsroom for a
small town. Across those four townships and one municipality you have a
ton of things that would need to be covered. The problem would not be a
lack of stories. The problem would be a lack of manpower to go find
stories to report.

It all comes down to money, I fear. This town wants to go neo-Amish.
Frankly that bit of sociocultural weirdness is a big problem I have yet
to figure out how to surmount.

More consideration is needed for all of this\ldots{}
