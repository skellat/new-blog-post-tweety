Well, we're in a predicament now. Warner Brothers Discovery is canceling
shows left and right while its division DC Studios is taking an axe to
its existing shows that it is producing. This is understandable only as
a tax dodge in that it is writing things off as total losses. They're
doing this in lieu of deep layoffs.

Throughout the media business as well as other industries there have
been fairly deep layoffs. Since March 2020 there has been a steady
acceleration in the number of broadcast radio stations going out of
business and surrendering their licenses to the Federal Communications
Commission. If the radio sounds like it has more static lately it is
because there quite literally are fewer stations on the air. Newspapers
are doing layoffs such as the big one that mega-chain Gannett just did a
few weeks ago. Even our local daily broadsheet \emph{The Star Beacon} is
looking metaphorically emaciated with an incredibly lower page count
than it had a few years ago.

The canary in the coal mine for how unhealthy our economy happens to be
is the change in television ads. If you start seeing ads for local
businesses disappear, ads for auto dealers disappear, and ads for things
like psychics in the vein of Miss Cleo increase then we're scraping the
bottom of the barrel for ad dollars at television stations. Whether we
like it or not the crazy political season acts as an indirect subsidy to
our radio and television broadcast stations as the rates charged to
candidates are the highest possible rates for ads, cannot be discounted,
and both sides pay the same rate. While there are political pundits
already lamenting that billions upon billions of dollars were spent
across the nation in ads this campaign season, the simple truth being
ignored is that all this cash has very likely been propping up many
broadcast stations financially. Now that many of these sources of
revenue are going away we may see closures accelerate again. The
political ads \emph{are not} being replaced by ads for other things far
too often which means we're seeing problems selling ad time.

Support your local media if you can. It may not be around that much
longer. Various bits of research expect one-third of newspapers to be
shuttered in the United States in the next three years. With the big
bankruptcy of iHeart in its ClearChannel days we saw that the massive
broadcast station conglomerates do have practical limits for just how
far they can grow before they're too big to function. Sinclair Broadcast
Group, Audacy, iHeart, and the like all have practical limits to where
they can't effectively own any more station properties. The mom and pop
operations are the ones dying off at the moment right now as well as,
oddly enough, campus radio stations at smaller colleges.

I wish I could say that replacing traditional broadcast with online
media would be an effective answer. That hasn't proved to truly be the
case. After a century of AM radio being around I don't think it can
really go anywhere.

Why do I care about any of this? The dying off of these media outlets
helps to fuel the radicalization that leads to MAGA and worse. If the
only place you can find the answer to questions about what's going on
around you is from the cultists on a social media platform then of
course radicalization is inevitable! Considering our low population
density in Ashtabula County we're seen as part of the \emph{Cleveland}
media market in some cases and in others we are part of no market at
all. The most popular radio station format here is Christian talk and if
you listen to it lately it sounds like a bunch of conspiracy theorists
wearing crucifixes rather than anything related to faith and belief.

Do I have the money to start my own broadcast station? No.~That would
cost on the order of a million dollars or more. As much as I look at
online alternatives I am not impressed with the usage numbers claimed.
This isn't a simple ``tech bro'' problem to fix but rather a nasty
socieconomic one.

Hopefully in 2023 some circumstances shift so opportunities become
available.
