Amy Howe writes in her blog in a post entitled
\href{https://amylhowe.com/2023/12/28/colorado-g-o-p-asks-justices-to-decide-trump-ballot-eligibility-issue-now/}{``Colorado
G.O.P. asks justices to decide Trump ballot eligibility issue now''} in
pertinent part:

\begin{quote}
\emph{Sekulow warned of ``catastrophic effects'' if the state supreme
court's decision is allowed to stand, predicting that ``any voter will
have the power to sue to disqualify any political candidate,'' which
will ``not only distort the 2024 presidential election but will also
mire courts henceforth in political controversies over nebulous
accusations of insurrection.''}
\end{quote}

\begin{quote}
\emph{Sekulow urged the court to take up the case quickly ``to prevent
the Colorado Supreme Court's decision from having an irreparable effect
on the electoral process.'' He indicated that he was ``prepared to abide
by whatever expedited processes this Court may set.''}
\end{quote}

One of the principles of the law in the United States is that it is not
supposed to have secret and arcane meanings. If you think you've found a
secret way to get out of taxes through some special interpretation of
history such as
\href{https://web.archive.org/web/20231228043521/https://history.house.gov/Historical-Highlights/1951-2000/The-admission-of-Ohio-as-a-state/}{Ohio
not being ``properly'' admitted as a state until the 1950s}, well let me
advise you that the federal government has a nice treatise posted
\href{https://www.irs.gov/privacy-disclosure/the-truth-about-frivolous-tax-arguments-introduction}{cataloging
all of these little bits of insanity and how they've been smacked down
in court}. That great list of bench-slaps is also available
\href{https://www.irs.gov/pub/irs-utl/2022-the-truth-about-frivolous-tax-arguments.pdf}{as
a 79 page PDF file} for your reading pleasure.

That's the problem with the arguments against disqualifying Donald Trump
under the third section of the 14th Amendment to the United States
Constitution. To say that he's not covered is to wrap yourself in knots.
Baude and Paulsen
\href{https://papers.ssrn.com/sol3/papers.cfm?abstract_id=4532751}{wrote
an article for the \emph{University of Pennsylvania Law Review}} that
lays out the case as to why he's covered. The pre-press PDF copy posted
to SSRN is only 126 pages. I got to hear Professor Paulsen speak to the
matter in an interview and I agree with his train of thought.

The decisions that have found against utilizing disqualification so far
have basically used odd linguistic dodges such as party primary not
being properly a candidacy, Donald Trump not actually being on the
ballot himself in a presidential contest but rather members of the
electoral college pledged to vote for him, and that only the vote by the
electoral college would be where he could be disqualified properly. All
of that is an absurd outcome based upon how the processes prior to the
electoral college works and the US Supreme Court decision in
\emph{Chiafalo v. Washington} essentially removes any free agency from
the electors nowadays. Constitutional provisions do not get thwarted by
mere enactments regardless of how clever they may be.

So, why am I afraid? The frivolous legal arguments that are popping up
to try to defend Trump are all part of a path that leads towards
violence. These sorts of arguments are connected with the
\href{https://web.archive.org/web/20231127075115/https://www.splcenter.org/fighting-hate/extremist-files/ideology/sovereign-citizens-movement}{capital
letters people}. The capital letters people have
\href{https://www.ncbi.nlm.nih.gov/pmc/articles/PMC7513757/}{grown in
popularity recently}.

I do hope this dies down. The odds are not in everyone's favor, it
seems.
