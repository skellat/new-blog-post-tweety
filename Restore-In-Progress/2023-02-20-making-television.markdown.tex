I did get some good news on Sunday relative to the streaming television
project. It is picking up further support in the community. Apparently
it is not just a random thing to spill out from my brain. There are
people who agree that this is actually needed.

Editions of the \href{https://www.starbeacon.com/}{\emph{Star Beacon}}
are getting even smaller. Weekday editions are getting down to
\textbf{twelve} broadsheet pages. There is plenty that happens locally
but without a functioning news media nobody knows about it.

Although people locally use Facebook like AOL was used back in the day
that might be in danger. Why? We have a convergence of actions leading
to possibly unintended consequences. Mark Zuckerberg
\href{http://web.archive.org/web/20230219174828/https://www.facebook.com/zuck/posts/pfbid02979GyAHwTKsMd7ngCiHTRCHyeTCEHwYe9Evq3YV2ffvxUY7fKVb9TGyKEUFBeo3kl}{announced
that Facebook will be charging \$12-15 per month for identity
verification, impersonation prevention, and access to customer support}
for both Facebook and Instagram. In comments to the post Mr.~Zuckerberg
mentions that identity verification is really expensive to conduct
manually which is why it has to be charged for and would not be done for
free.

Separately from that we have the executive budget proposal to the
legislature here in Ohio
\href{http://web.archive.org/web/20230218064633/https://governor.ohio.gov/administration/lt-governor/020823}{proposing
something called the ``Social Media Parental Notification Act''}. In the
proposed law under threat of fines and possible suspension of service
within the bounds of the stat, the law would require certain social
platform companies to: ``Create a method to determine whether the user
is a child under the age of 16''. The proposal effectively requires
identity verification of \textbf{all Ohioans} before they use social
media platforms so that the state government ensures no children under
the age of 16 use these things without parental consent. The proposal
provides no funding to the companies to carry out these functions.

If enacted by Ohio the result would be that Facebook would become either
paid-only or simply unavailable here. I'm not sure rolling the dice on
litigation would be the safest move in the current times. Judges out
here have done crazy unexpected things that could be best termed as
``weird''.

My state is run by crazy people.

At least none of them have one of the newfangled
\href{https://web.archive.org/web/20230219092604/https://www.theverge.com/2023/2/16/23602833/meta-instagram-channels-telegram-facebook-messenger}{``Meta
Channel''} things yet\ldots{}
