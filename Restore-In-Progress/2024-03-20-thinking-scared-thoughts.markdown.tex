Well, when you have a major presidential candidate acting like a crazy
person you wind up with feelings of being scared. I think those feelings
may be justifiable. After all, during a speech over the prior weekend we
had the ex-president and leader of the out-party loudly talking about
elections \emph{that never happened}, calling convicted felons patriots,
and otherwise sounding like an escapee from a psychiatric care
institution.

The local Republican Party picked up a bunch of new partisans in the
primary election on Tuesday. That major presidential candidate got a
relatively large amount of the vote. Quite a lot of people are drinking
the kool-aid.

In times like this I recognize that my programming abilities are not up
to snuff. I can handle LaTeX fairly decently. What I am looking to have
done isn't handled in LaTeX, though.

There exist scripts like
\href{https://github.com/rss2email/rss2email}{rss2email} to take the
output of an RSS feed and convert it into e-mail messages to be plopped
into your e-mail account. I found an
\href{https://github.com/cordawyn/rss2imap}{rss2imap} variant that
somebody made in Haskell that simply plops things directly to the IMAP
server for you to read.

What I am more interested in is having things archived. As we're in the
midst of a presidential campaign the likelihood of active information
operations by adversarial powers is pretty much assured. I want
something akin to rss2email or rss2imap that takes the item, converts
its source page to \href{https://specs.webrecorder.net/wacz/1.1.1/}{WACZ
format}, and then includes that somewhere it can be accessed in a
systematic fashion. For bonus points, I would definitely want something
akin to
\href{https://en.wikipedia.org/w/index.php?title=El_Paquete_Semanal&oldid=1211054142}{El
Paquete Semanal} where a single WACZ file could be built per day
containing all the observed RSS items seen that day with an appropriate
index.

How would I do something like this? Ideally I would want the software to
build data packages like that running at least on a VPS located
\emph{outside the geographical boundaries of the United States}. I would
want to build multiple routes to access that VPS including
\href{http://www.nncpgo.org/}{NNCP} and probably
\href{https://web.archive.org/web/20230927022215/https://www.nutsvolts.com/magazine/article/may2015_Steber}{over
the air too}.

Why do something like this? The Steve Bannon strategy of flooding the
zone with crap is going to be going into effect pretty soon. People are
going to have a very hard time telling what is ral and what is not. All
the generative artificial intelligence stuff out there is only going to
accelerate the impact of the Bannon strategy. It'll easily impact social
cohesion in so many bad ways.

I guess I'll need to look at hiring somebody when funds become freed up.
That'll be some day off in the future, alas.
