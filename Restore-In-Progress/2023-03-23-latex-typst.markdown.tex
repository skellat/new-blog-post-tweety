Consider the following source file that we'll call
\texttt{passthrough.markdown}:

\begin{Shaded}
\begin{Highlighting}[]
\NormalTok{There is a project trying to build something like LaTeX but isn\textquotesingle{}t LaTeX that can be found at: }\OtherTok{\textless{}https://github.com/typst/typst/\textgreater{}}

\NormalTok{It requires some review.}
\end{Highlighting}
\end{Shaded}

Then consider the following source file that we'll call
\texttt{markdown-test.tex}:

\begin{Shaded}
\begin{Highlighting}[]
\CommentTok{\% !TeX program = lualatex}
\CommentTok{\% !TeX encoding = UTF{-}8}

\BuiltInTok{\textbackslash{}documentclass}\NormalTok{[12pt,letterpaper]\{}\ExtensionTok{extarticle}\NormalTok{\}}
\BuiltInTok{\textbackslash{}usepackage}\NormalTok{[english]\{}\ExtensionTok{babel}\NormalTok{\}}
\BuiltInTok{\textbackslash{}usepackage}\NormalTok{\{}\ExtensionTok{graphicx}\NormalTok{\}}
\BuiltInTok{\textbackslash{}usepackage}\NormalTok{\{}\ExtensionTok{relsize}\NormalTok{\}}
\BuiltInTok{\textbackslash{}usepackage}\NormalTok{\{}\ExtensionTok{markdown}\NormalTok{\}}
\BuiltInTok{\textbackslash{}usepackage}\NormalTok{\{}\ExtensionTok{hyperref}\NormalTok{\}}
\BuiltInTok{\textbackslash{}usepackage}\NormalTok{\{}\ExtensionTok{xurl}\NormalTok{\}}
\BuiltInTok{\textbackslash{}usepackage}\NormalTok{[autostyle, english = american]\{}\ExtensionTok{csquotes}\NormalTok{\}}
\BuiltInTok{\textbackslash{}usepackage}\NormalTok{\{}\ExtensionTok{fontspec}\NormalTok{\}}
\FunctionTok{\textbackslash{}setmainfont}\NormalTok{\{Domitian\}}
\FunctionTok{\textbackslash{}setsansfont}\NormalTok{\{Quattrocento Sans\}}
\FunctionTok{\textbackslash{}setmonofont}\NormalTok{\{IBM Plex Mono\}}

\BuiltInTok{\textbackslash{}usepackage}\NormalTok{\{}\ExtensionTok{lastpage}\NormalTok{\}}

\FunctionTok{\textbackslash{}MakeOuterQuote}\NormalTok{\{"\}}
\BuiltInTok{\textbackslash{}usepackage}\NormalTok{\{}\ExtensionTok{microtype}\NormalTok{\}}

\FunctionTok{\textbackslash{}DeclareGraphicsExtensions}\NormalTok{\{.pdf,.png,.jpg\}}

\FunctionTok{\textbackslash{}author}\NormalTok{\{\}}
\FunctionTok{\textbackslash{}title}\NormalTok{\{\}}
\FunctionTok{\textbackslash{}date}\NormalTok{\{\}}

\FunctionTok{\textbackslash{}hypersetup}\NormalTok{\{}
\NormalTok{    pdftitle=\{\},    }
\NormalTok{    pdfauthor=\{\},   }
\NormalTok{    pdfsubject=\{\},  }
\NormalTok{    colorlinks=false,       }
\NormalTok{    pdfpagemode=UseOutlines,}
\NormalTok{    bookmarksopen=true,}
\NormalTok{    pdfkeywords=\{\},}
\NormalTok{    allcolors=\{black\},}
\NormalTok{    pdfstartview=FitH,}
\NormalTok{\}}

\KeywordTok{\textbackslash{}begin}\NormalTok{\{}\ExtensionTok{document}\NormalTok{\}}
\FunctionTok{\textbackslash{}markdownInput}\NormalTok{\{passthrough.markdown\}}
\KeywordTok{\textbackslash{}end}\NormalTok{\{}\ExtensionTok{document}\NormalTok{\}}
\end{Highlighting}
\end{Shaded}

If you take those sources, make them proper local files, and then run
them through the latest \href{https://tug.org/texlive/}{TeX Live} you'll
find that the Markdown file will get incorporated into the LaTeX source
and still be processed thanks to the
\href{https://ctan.org/pkg/markdown}{markdown} package on CTAN currently
maintained by Vít Novotný. This is somewhat different from using
\href{https://pandoc.org}{pandoc}. Consider the following invocation of
\texttt{pandoc} using the Markdown source from above:

\begin{Shaded}
\begin{Highlighting}[]
\ExtensionTok{pandoc} \AttributeTok{{-}{-}output}\NormalTok{ contrast.tex }\AttributeTok{{-}{-}from}\NormalTok{ markdown }\AttributeTok{{-}{-}to}\NormalTok{ latex }\AttributeTok{{-}{-}standalone}\NormalTok{ passthrough.markdown}
\end{Highlighting}
\end{Shaded}

That ends up giving you a \texttt{contrast.tex} that looks different
than what I created above. The difference was a bit subtle. The link in
the markdown snippet isn't turned into something that would generate a
hyperlink in the LaTeX. In the first the \texttt{markdown} package reads
the link in the incorporated-by-reference source and will turn it into a
hyperlink if you compile the LaTeX source. That's not a difference you
would expect.

Why worry about this? I haven't fiddled with
\href{https://typst.app}{typst} yet. I just wonder how flexible it may
be as well as how consistent.

Besides, CTAN has an entire category devoted to
\href{https://ctan.org/topic/markup}{parsing alternative markup formats}
which is why I have trouble understanding the need for something like
typst. Much of what it apparently does is seemingly already implemented.
Was the point to that \href{https://github.com/typst/typst}{they
implemented their product in Rust} in contrast to the unique language
LaTeX is written in?

Yeah, I can write technical posts from time to time. I have MacTex 2023
installed on my MacBook Pro M1 now. I enjoy making good-looking
documents using \emph{proper} tools.
