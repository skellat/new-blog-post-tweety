I've been considering a bit of a hang-up that I have. This comes out of
the string of medical appointments that I've had since the start of the
month. Yes, this relates in part to \emph{The Condition} and it is
somewhat in a weird way.

I am short according to the various tables of ``average'' heights in the
USA. Where in the world might I be someone of average height? Apparently
that would includes places like:

\begin{itemize}
\tightlist
\item
  Turkmenistan
\item
  Trinidad and Tobago
\item
  Spain
\item
  Jordan
\item
  Moldova
\item
  Dominican Republic
\item
  People's Republic of China
\item
  Singapore
\item
  Poland
\end{itemize}

Yeah, I doubt that
\href{https://en.wikipedia.org/w/index.php?title=Average_human_height_by_country&oldid=1217241037}{table
on Wikipedia} some days. In this instance it \emph{seems} to be right.
I'll roll with it.

Yes, my height is below average for a man in the United States. There
were school kids bigger than me at a restaurant that I visited with
family on Sunday after church. I know my height is below average but
\emph{seriously} I feel like a halfling some days.

My height would be considered tall for a woman in the United States. A
recent struggle has been trying to figure out why one of The Specialist
Doctors keeps addressing me as a woman. Is it due to my height and not
having a basso profundo voice? That particular one of The Specialist
Doctors likely has some cultural baggage of their own that they're
looking at me through, I think.

Why the worries about height? I think it ties in to the weight issue
that was being discussed with one of the other doctors. When it comes to
dealing with The Condition there are some paradoxes to deal with. Being
overweight but having a hemoglobin A1C around 4 means that high blood
sugar is not contributing to the obesity. It also means that Popular
Injectible Drug is highly contraindicated for use and you end up reading
papers about \{REDACTED\} and \{REDACTED\} that very few people would
ever read about since the Orphan Drug Act would end up getting invoked.

That brings me back to the body weight problem. I've sat down with the
clinical dietitian and reviewed what it would take to get things under
control. Why am I not succeeding getting things under control?

I think this comes down to environmental factors. A man my age in my
local community generally has a BMI of 30+. Even if I am shorter than
the average man in my local community I am likely unconsciously trying
to fit in. At my height at a BMI of 24.9 based upon the way things look
in morph software I would look somehow more alien than I already do in
my community. The bias towards conformance is pretty strong as well as
the push to ``look my age'' though the health complications that come
from trying to do that outweigh the benefits, it seems.

On Tuesday there was a story in \emph{Ohio Capital Journal} by Susan
Tebben talking about
\href{https://web.archive.org/web/20240416114129/https://ohiocapitaljournal.com/2024/04/16/mental-health-care-costs-top-problems-for-ohioans-in-new-study/}{the
really bad healthcare situation here in Ohio}. What makes it sad is how
many world class hospitals are located here in Ohio! The default in
communities like Ashtabula is to be unhealthy and for adults to have
BMIs of 30+. That could also be why we have such an unhealthy community
that
\href{https://web.archive.org/web/20240105093608/https://www.unionleader.com/news/health/how-red-state-politics-are-shaving-years-off-american-lives/article_a15969c1-6959-526b-91d2-9d1c966c15fd.html}{even
the \emph{Washington Post} syndicates stories about how strangely this
county is screwed up}.

Now that I know what the hang up is I have to figure out what to do
about it. Again, the popular class of Popular Injectible Drug are found
to not work in people with The Condition and research is in progress on
options for people with The Condition. Until then I just have to
remember that my caloric needs are far less than the recommended average
seen on nutrition labels. Making change isn't nearly as simple as it may
seem.

Of course, it doesn't help that people with The Condition have weight
problems and it is even seen as a diagnostic symptom. As someone who
processes insulin in a hyper-efficient manner I have to be careful in
how this whole issue of weight management gets approached.
