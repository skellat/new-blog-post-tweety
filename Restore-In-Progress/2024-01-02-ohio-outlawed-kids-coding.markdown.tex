I wish I were sensationalizing this. The Social Media Parental
Notification Act is becoming effective here as of January 15th. It
targets ``social media operators''. Does it only target traditional
sites like Facebook and Instagram? Heck no! Here's how it defines a
social media operator as seen in the Ohio Attorney General's
\href{https://web.archive.org/web/20231230000932/https://www.ohioprotects.org/faq}{own
FAQ}:

\begin{quote}
An operator means anyone who operates an online website, service or
product that allows users to do all of the following:

\begin{enumerate}
\def\labelenumi{\arabic{enumi}.}
\tightlist
\item
  Interact socially with other users on the website, service or product.
\item
  Create a public or semipublic profile to sign into and use the
  website, service or product.
\item
  Populate a list of other users with whom they share a connection
  within the website, service or product.
\item
  Create or post content viewable by others -- for example, on message
  boards, chat rooms, video channels, direct messages, or a main feed
  that contains content generated by others on the website, service or
  product.
\end{enumerate}

Note that this law applies only to operators of an online website,
service or product that targets children or is reasonably anticipated to
be accessed by children. This law does not apply to e-commerce websites
that allow for posting of reviews or to media outlets that report the
news.
\end{quote}

As the fools in the legislature have this written, use of sites like
Launchpad, sr.ht, Gitlab, and GitHub would seemingly hit all four
points. The first point definitely can be met by filing issues and bug
reports on other people's repositories. The second point is met by
setting up your profile on one of those sites and likely making the hash
of a signing key visible. The third point is met quite easily by being
part of a changelog that lists contributors to a release. The fourth
point is met by pushing code to public repositories.

This applies to children 16 and younger. Conceivably sr.ht, Gitlab,
GitHub, and Launchpad really need to geo-block Ohio for their own safety
due to this
\href{https://web.archive.org/web/20240101080348/https://codes.ohio.gov/ohio-revised-code/section-1349.09}{poorly
conceived and written law} that includes massive financial penalties.
Source code hosting would most likely \emph{not} meet the exception for
cloud storage and cloud computing services. As that's been defined in
the Ohio Revised Code, that refers more to things like Microsoft
OneDrive and Microsoft Windows 365 cloud PC.

Unintended consequences? Of course they're unintended. Nobody's going to
amend the statute, though.
