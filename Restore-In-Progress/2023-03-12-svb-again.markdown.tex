There has been chatter online about possible ripple effects from the
collapse of Silicon Valley Bank. To repeat a statement made by Bill
Ackman on Twitter at
\url{https://twitter.com/BillAckman/status/1634564398919368704}, he
said:

\begin{quote}
\emph{The gov't has about 48 hours to fix a-soon-to-be-irreversible
mistake. By allowing @SVB\_Financial to fail without protecting all
depositors, the world has woken up to what an uninsured deposit is ---
an unsecured illiquid claim on a failed bank. Absent @jpmorgan @citi or
@BankofAmerica acquiring SVB before the open on Monday, a prospect I
believe to be unlikely, or the gov't guaranteeing all of SVB's deposits,
the giant sucking sound you will hear will be the withdrawal of
substantially all uninsured deposits from all but the `systemically
important banks' (SIBs). These funds will be transferred to the SIBs, US
Treasury (UST) money market funds and short-term UST. There is already
pressure to transfer cash to short-term UST and UST money market
accounts due to the substantially higher yields available on risk-free
UST vs.~bank deposits. These withdrawals will drain liquidity from
community, regional and other banks and begin the destruction of these
important institutions. The increased demand for short-term UST will
drive short rates lower complicating the @federalreserve's efforts to
raise rates to slow the economy. Already thousands of the fastest
growing, most innovative venture-backed companies in the U.S. will begin
to fail to make payroll next week. Had the gov't stepped in on Friday to
guarantee SVB's deposits (in exchange for penny warrants which would
have wiped out the substantial majority of its equity value) this could
have been avoided and SVB's 40-year franchise value could have been
preserved and transferred to a new owner in exchange for an equity
injection. We would have been open to participating. This approach would
have minimized the risk of any gov't losses, and created the potential
for substantial profits from the rescue. Instead, I think it is now
unlikely any buyer will emerge to acquire the failed bank. The gov't's
approach has guaranteed that more risk will be concentrated in the SIBs
at the expense of other banks, which itself creates more systemic risk.
For those who make the case that depositors be damned as it would create
moral hazard to save them, consider the feasibility of a world where
each depositor must do their own credit assessment of the bank they
choose to bank with. I am a pretty sophisticated financial analyst and I
find most banks to be a black box despite the 1,000s of pages of @SECGov
filings available on each bank. SVB's senior management made a basic
mistake. They invested short-term deposits in longer-term, fixed-rate
assets. Thereafter short-term rates went up and a bank run ensued.
Senior management screwed up and they should lose their jobs. The
@FDICgov and OCC also screwed up. It is their job to monitor our banking
system for risk and SVB should have been high on their watch list with
more than \$200B of assets and \$170B of deposits from business
borrowers in effectively the same industry. The FDIC's and OCC's failure
to do their jobs should not be allowed to cause the destruction of
1,000s of our nation's highest potential and highest growth businesses
(and the resulting losses of 10s of 1,000s of jobs for some of our most
talented younger generation) while also permanently impairing our
community and regional banks' access to low-cost deposits. This
administration is particularly opposed to concentrations of power.
Ironically, its approach to SVB's failure guarantees duopolistic banking
risk concentration in a handful of SIBs. My back-of-the envelope review
of SVB's balance sheet suggests that even in a liquidation, depositors
should eventually get back about 98\% of their deposits, but eventually
is too long when you have payroll to meet next week. So even without
assigning any franchise value to SVB, the cost of a gov't guarantee of
SVB deposits would be minimal. On the other hand, the unintended
consequences of the gov't's failure to guarantee SVB deposits are vast
and profound and need to be considered and addressed before Monday.
Otherwise, watch out below.}
\end{quote}

Frankly nobody has a clue where things go from here. Things may be
totally fine on Monday. The bottom could also fall out on our economy
here in the USA too due to \emph{further} bank runs caused by social
media-fueled disinformation/misinformation going viral.

What bookie in Vegas would give odds on this? There's no way to
reasonably bet on the outcome to this\ldots{}
