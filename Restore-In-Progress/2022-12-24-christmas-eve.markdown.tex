This is more intended to log thoughts than to be a coherent essay.

I know I can use PowerPoint to do this. I did \emph{that} for one of the
local cablecast channels for two years. Hooking up a Raspberry Pi zero
and using LibreOffice to push a repeating slideshow would not be too
hard but for the problem of updating the slideshow.

I had been considering the notion of using a modified version of
\href{http://web.archive.org/web/20221222032104/https://feh.finalrewind.org/}{feh}
that was not from the package archives. That could be used to set up a
slideshow. I hadn't had the chance to test it yet but if I understood
the mechanics correctly I could have a fixed set of filenames in a
directory that feh would flip through then I could use sftp to replace
those files dynamically as needed. The filenames would be something like
``first'', ``second'', ``third'', etc.

Technically the ``video bulletin board'' that I referred to yesterday is
more technically referred to as non-interactive in-vision teletext
display. At least \href{https://github.com/peterkvt80/vbit-iv}{one
GitHub repository} exists with software to decode teletext that could
then be piped into OBS Studio or other software and then sent downstream
to a streaming service. This requires further investigation.
