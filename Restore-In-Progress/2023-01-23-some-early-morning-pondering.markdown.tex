I sent a Letter To The Editor to \emph{The Gazette} about the streaming
situation locally. We have things like
\href{https://vimeo.com/user126359532}{Conneaut CAT TV},
\href{https://www.youtube.com/channel/UCY_bWD4nXv5U1FJZF3tfZ6g}{Eagleville
Bible Church streaming}, and
\href{https://www.twitch.tv/genevachurchofchrist}{Geneva Church of
Christ streaming} already. We need a general streaming television
service to cover \emph{all} of Ohio's largest rural county. We have the
talent. We have the tech, relatively speaking. We lack funding and
production space. With a ``Letter To The Editor'' the situation might
get placed in front of eyeballs who might take an interest in the
matter. Strangely enough I do have some sort of a readership locally
when I manage to remember to write things for publication in \emph{The
Gazette}.

The list of radio stations on Wikipedia found at
\url{https://en.wikipedia.org/wiki/Template:Ashtabula_Radio} shows many
stations that do not have local air talent routinely broadcasting live.
Breaking it down in list form based upon the infobox on Wikipedia:

\begin{itemize}
\tightlist
\item
  KEC58 is NOAA Weather Radio which is computer-generated speech
  programmed out of the National Weather Service forecast office in
  Cleveland
\item
  KZZ32 is NOAA Weather Radio which is computer-generated speech
  programmed out of the National Weather Service forecast office in
  Cleveland
\item
  W210DP is a low-power FM translator situation on top of a residential
  high-rise in downtown Ashtabula City. W210DP relays the signal of NPR
  affiliate WYSU which is owned by Youngstown State University
\item
  WCVJ acts as a translator of the Air1 service of Educational Media
  Foundation and there is no local air talent
\item
  WFUN has no local air talent as it runs ESPN Radio 24/7
\item
  WFXJ-FM has a classic rock DJ who is
  \href{https://en.wikipedia.org/wiki/Voice-tracking}{``voice tracked''}
  and is not actually present live for the few hours of his shift while
  the rest of the time is covered by syndicated programming
\item
  WGOJ acts generally as a translator for Bible Broadcasting Network and
  provides syndicated programming when there is nothing from BBN
\item
  WKKY has DJs working live although it does sound like they've shifted
  partially to some voice-tracking and they do bring in \emph{some}
  syndicated programming on the country formatted station
\item
  WKSV is a translator for WKSU, the flagship station operated by
  Ideastream yet owned by Kent State University and is yet another NPR
  affiliate
\item
  WMIH is a low-power translator for the
  \href{https://thestationofthecross.com/}{Stations of the Cross
  network} which provides ``Catholic Radio'' out of their headquarters
  in New York State
\item
  WOHK is a translator for Educational Media Foundation's K-Love network
  has no local staff in Ashtabula County
\item
  WVMU is a translator for
  \href{https://www.moodyradio.org/stations/cleveland}{Moody Radio
  Cleveland} and has no local staff in Ashtabula County
\item
  WREO-FM works out of studios in Mentor with occasional live shifts but
  mostly voice-tracking and some syndicated programming with the focus
  shifted toward Lake County even though the transmitted is located in
  Ashtabula County out on Jefferson Road
\item
  WQGR is an Oldies station that is voice-tracked with syndicated
  programming too while being based out in Madison
\item
  WWOW is based in Conneaut and airs mostly syndicated programming but
  occasionally airs local content that is focused on Conneaut
\item
  WYBL is a country-formatted station that is voice-tracked and shares
  air talent with other Media One stations such as WFXJ
\item
  WZOO-FM is a ``Classic Hits'' station with voice-tracked air talent
  and syndicated programming
\end{itemize}

There is not really anything like the local radio of old still in
existence here in Ashtabula County. If people need prompt weather
warnings they truly need a radio capable of receiving NOAA Weather Radio
broadcasts. The AM/FM broadcast stations likely do not have anybody at
home at the ``stations''. Some of those stations have no local studios
and are just secured equipment cabinets in blockhouses in fields
translating signals from elsewhere.

Yes, producing content is expensive. There is a need for \emph{good}
content to serve local information needs. You can get plenty of crap for
free on Facebook that can prove rather useless. In a world where QAnon
still exists as well as other violent groups that organize online there
remains a need for communities like my own to have ways to promptly
report the news and keep each other in the loop. The
\href{https://www.starbeacon.com/}{\emph{Star Beacon}} is getting
thinner and thinner in its print page count. The weeklies from Gazette
Newspapers only come out once per week. With no real human attention
from the radio stations and our ``local'' television stations having
such huge territories to cover we got lost in the din.

What I don't want to do is hit the point of giving up due to
hopelessness. That's incredibly easy to do in this town. Step by step we
may yet see some positive change in this area, I hope.
