Keeping the lines of communication open will be a big goal this year.
That may involve some of these tool sets:

\begin{itemize}
\tightlist
\item
  \href{https://web.archive.org/web/20240112005127/https://git-scm.com/book/en/v2/Git-Tools-Bundling}{Using
  git bundle}\\
\item
  Wrapping my head around
  \href{https://web.archive.org/web/20231221173431/https://drewdevault.com/2018/07/02/Email-driven-git.html}{what
  an e-mail based workflow for git looks like} and how
  \href{https://web.archive.org/web/20231219095927/https://begriffs.com/posts/2018-06-05-mailing-list-vs-github.html}{some
  people leverage git via e-mail} perhaps\\
\item
  Learning about
  \href{https://web.archive.org/web/20231225131129/http://bugseverywhere.org/}{Bugs
  Everywhere}
\item
  Handling the \href{https://git-send-email.io/\#step-1}{git via e-mail}
  tutorial
\item
  What
  \href{https://web.archive.org/web/20230324145621/https://www.wikihow.com/Install-and-Set-Up-Free-to-Air-Satellite-TV-Program-Receiver-System}{Free
  To Air satellite television} looks like from the US perspective
\item
  Properly automating WEFAX
  \href{https://www.youtube.com/watch?v=l1QyDVgclto}{reception} (No, I
  don't know why NOAA thought posting a link to a third-party
  instructional video
  \href{https://web.archive.org/web/20231123151628/https://www.weather.gov/marine/radiofax_charts}{was
  a great idea} relative to
  \href{https://web.archive.org/web/20240129060301/https://www.weather.gov/media/marine/rfaxatl.txt}{getting
  charts})
\item
  And for when eyeball vision becomes a potentially troublesome thing
  again, consideration of how a modern
  \href{https://web.archive.org/web/20231223180025/https://drewdevault.com/2019/10/30/Line-printer-shell-hack.html}{TTY
  clone} or
  \href{https://web.archive.org/web/20230610191232/https://hackaday.com/2019/11/14/upgrade-board-turns-typewriter-into-a-teletype/}{typewriter
  retrofit} could be done so as to keep on truckin' even though there
  would be a huge problem of Unicode compliance
\end{itemize}

These are all obscure and arcane things. My life goes down obscure and
arcane roads, though. I learn that the hard way when I go to visit the
doctor. Normal people have doctors that complain about patients
consulting ``Doctor Google'' looking up symptoms. In my instance I'm
reading pretty much the same journal articles my doctors are seeing from
National Library of Medicine about The Rare Condition and finding an
oddball thing at a visit could result in being the subject of a journal
article at this rate.

I am on a journey of lifelong learning, it seems.
