As much as I like screen captures, I will dispense with showing them in
this post, I think. Quoting @kissane@mstdn.social:
\url{https://mstdn.social/@kissane/110153372803611489}
\href{\%7B\%7Bsite.url\%7D\%7D/img/bring-creators-to-masto.jpg}{\#retoot}:

\begin{quote}
\emph{Post by Erin Kissane. ``Right now, a lot of fedi advocates are
asking authors and artists to leave the social media sites that have
allowed them to get by, but without doing much/anything to support those
newcomers once they arrive here---even leaving aside the old-guard
pushback against ``self promotion.'' And idk, ``staying on twitter for
your work makes you the object of my disgust,'' is maybe not the
persuasive tool a lot of people here seem to believe it is. 🤷🏻'' Posted
on Apr 6, 2023 at 14:51}
\end{quote}

We must contrast that with another post though. Quoting
@w7voa@journa.host: \url{https://journa.host/@w7voa/110155351634818818}
\href{\%7B\%7Bsite.url\%7D\%7D/img/w7voa-on-npr-going-silent-on-twitter.jpg}{\#retoot}:

\begin{quote}
\emph{Post by Steve Herman. ``Since being labeled ``US state-affiliated
media'' by \#Twitter, NPR has not tweeted.
npr.org/2023/04/06/1168455846/\ldots'' The post contains an image with
no description. Posted on Apr 6, 2023 at 23:14}
\end{quote}

The post Steve Herman linked to can be found at
\url{https://www.npr.org/2023/04/06/1168455846/elon-musk-says-nprs-state-affiliated-media-label-might-not-have-been-accurate}.
The picture in the post shows NPR's account page on Twitter bearing the
label ``US state-affiliated media'' which is akin to the labels
generally applied to government propaganda outlets like RT and Sputnik
out of Russia. Oddly enough, this label was applied to the private
non-profit NPR \emph{but} it was not applied to Voice of America which
is actually owned by the United States Agency for Global Media and has
employees that are federal civil servants.

And then there was this, I guess. Quoting @dangillmor@mastodon.social:
\url{https://mastodon.social/@dangillmor/110157692641473447}
\href{\%7B\%7Bsite.url\%7D\%7D/img/gillmor-on-capitulation.jpg}{\#retoot}:

\begin{quote}
\emph{Post by Dan Gillmor. ``So NPR has given its answer to the Musk
regime's loud contempt for the news outlet: To no one's surprise, NPR
utterly capitulated. The NPRPolitics account -- and, apparently, other
sub-accounts -- is posting as usual on Twitter. This supports Musk's
business. Again, remember some relevant history. NPR was one of the most
ardent purveyors of false balance, and normalizers of extremism, after
the 2016 election. Anyone who imagined the organization to find a spine
now was just naive.'' Posted on Apr 7, 2023 at 09:09}
\end{quote}

My feelings on Twitter are very mixed. Erin Kissane is right that the
old-guard pushback is very much still there and is not likely to leave.
There aren't counterparts yet in the fediverse to the economic
structures creators possess on Twitter. Twitter is having problems with
an arbitrary leader, though. Mediaite reports
\href{https://web.archive.org/web/20230406081155/https://www.mediaite.com/news/elon-musk-appears-covered-w-in-twitter-sign-hq-days-after-changing-app-logo-doge-meme/}{its
headquarters signage has been modified to now say Titter}. Semafor
\href{https://web.archive.org/web/20230406183101/https://www.semafor.com/article/04/05/2023/twitter-falls-short-in-policing-russian-and-chinese-state-backed-media}{reports
that Twitter is letting propaganda content from the Russian Federation
and People's Republic of China apparently run wild}. The Verge reports
\href{https://web.archive.org/web/20230407033126/https://www.theverge.com/2023/4/6/23673043/twitter-substack-embeds-bots-tools-api}{that
Twitter cuts off content creators on Substack from embedding tweets}.

I'm not sure Twitter should still be functioning. We're in a ``Weekend
At Bernie's'' situation. It is dead. We're parading the carcass around
while trying to act like everything is okay. People are trying to keep
on keeping on but the site has become a den of misinformation and
disinformation. It is now effectively a battlefield for information
warfare/psychological warfare rather than a communications platform.

In the end, stay if you have to in the bad neighborhood. After all, the
fellow who is now The Owner of Twitter is degenerating into a bumbling
\href{https://en.wikipedia.org/w/index.php?title=Boss_Hogg&oldid=1148453036}{Boss
Hogg} sort of character before our eyes and is not taking very good care
of the place. We must all work together to build a better world if we
want a transition to succeed. Sticking with the status quo is going to
lead to compromises that we can't afford in the future.
