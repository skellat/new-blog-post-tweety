Alrighty, we start first with \emph{1d100 Spaceship Missions} by Terry
Herc. We'll roll using Dice by PCalc.

We get a 21. That gives us this result:

\begin{quote}
\emph{Install a deep-space telescope in an unexplored region of space.
The first images reveal an approaching, unknown massive cosmic entity,
sparking a sector-wide alarm. 125kcr}
\end{quote}

We turn to \emph{Deep Space Backwater Planetoid Generator} by D10
Dimensions. We'll roll using Dice by PCalc. There are many multiple
rolls to be made. There are nine tables to be worked through.

The first throw is 1d20 + 3d10. We get a 2 on the 1d20 and the 3d10
gives us 9, 8, and 4. That gives us this result:

\begin{quote}
\emph{Hector OR Cassius OR Haru OR Chad}
\end{quote}

And I will make a choice from those four which gives us:

\begin{quote}
\emph{Hector 984}
\end{quote}

Our second throw will give us an orbital profile for Hector 984. We roll
a 1d20 and get a 14. It gives us this result:

\begin{quote}
\emph{Small asteroid orbits it (micro-moon)}
\end{quote}

Our third throw gives us features for Hector 984. We roll a 1d20 and get
17. That gives us this result:

\begin{quote}
\emph{Covered in broad, flat mild plains}
\end{quote}

Our fourth throw is on the optional metals and minerals table. We roll
2d20 and get 3 as well as 11. That gives us these results:

\begin{quote}
\emph{Sulphur - Common but useful ore} \emph{Crystals - Naturally formed
in vacuum}
\end{quote}

This place is already not looking like a vacation spot.

Our fifth throw is on the lifeforms table. We roll 1d20 and get a 2 even
though we almost got a 20. That gives us this result:

\begin{quote}
\emph{No Life / Foreign life visited \& then left}
\end{quote}

Our sixth throw is on the signs of civilization table. We roll up to 1d3
times on this table. We first roll a 1d6, get a 2, and then divide that
by 2 to get 1. This means that we will only get to roll 1d20 on this
table. We roll 1d20 and get 6 which gives us this result:

\begin{quote}
\emph{Industrial Factory and Company Town}
\end{quote}

Our seventh throw is a 1d10 to determine the planetoid's atmosphere. We
roll the 1d10 and get a 4. That result gives us this cue:

\begin{quote}
\emph{Normal - Supports Life.}
\end{quote}

Our eighth throw is a 1d10 to determine the planetoid's size and
gravity. We roll the 1d10 and get a 9. That result gives us this cue:

\begin{quote}
\emph{(126 + 5d6 x 100) KM / 1.2 Gravity}
\end{quote}

Our ninth throw is a 1d10 to determine the planetoid's place in the
universe. We roll the 1d10 and get a 8. That result gives us this cue:

\begin{quote}
\emph{Tourist destination for old civilization}
\end{quote}

Now we're starting to get somewhere, I think.

We turn to \emph{1d100 Lost Transmissions} by Terry Herc and roll 1d100
three times. We get 4, 17, and 27. Those numbers give us these results:

\begin{quote}
\emph{{[}Distressed voice{]} ``This is the cruiser Radiant Dawn. We've
found an alien device on Xedros-3. It's started\ldots reacting. I'm no
scientist, we need a specialist here, stat!''}

\emph{``Buying phase crystals. Offering double the galactic rate. Meet
at the Trade Spire on Lyrion-4.''}

\emph{``Anyone near the Draco Nebula, this is the Golden Falcon. We've
found a driftiing colony ship. It's\ldots{} old, like pre-hyperdrive
old. There are cryo-pods. We need help with salvage and rescue.''}
\end{quote}

We then turn to \emph{1d100 Spaceships and Captains} by Terry Herc then
throw a 1d100 again. We get a 30 which gives us the following result:

\begin{quote}
\emph{Galaxy Grinder: Captain Dex Bolaris, an erstwhile gladiatorial
champion, now known for his daunting presence and fearless nature. His
ship, the Galaxy Grinder, is a former warship retrofitted into a combat
sports arena, hosing thrilling intergalactic tournaments.}
\end{quote}

We then turn to \emph{200 Space Civilization Occupations} by D10
Dimensions and throw d100 three times. That gives us 43, 66, and 31
which translates to these results:

\begin{quote}
\emph{Cyber Surgeon}

\emph{Laborer Overseer}

\emph{Bot Mechanic}
\end{quote}

We then throw d100 two more times. That gives us 72 and 7 which
translates to:

\begin{quote}
\emph{Comms Tower Manager}

\emph{Space Pirate}
\end{quote}

Well, we now have a scene, characters, and mission. This gives us a
start to building a story. We'll actually start writing that story at a
later time.
