There is a way to get news programming launched even with the space
limitation. I know I said in \emph{The Gazette} that remote production
techniques wouldn't work here. To the extent that related to producing a
\emph{traditional} video program that is true. Going down a
\emph{non-traditional} path may make it possible.

Again, consider what is being done by Conneaut CAT TV with
\href{https://vimeo.com/799131311}{Hometown Happenings} in its most
recent edition. Martha is reading a script on-screen. Could I work with
a narrated \emph{quick} slideshow to provide a headlines service? It
would not be optimal but it \emph{could} be accomplished within the
limits of available resources.

\href{https://support.apple.com/guide/keynote/record-presentations-tan81813d552/mac}{Keynote
from Apple} will export slideshows with narration to video. It appears
that
\href{https://support.microsoft.com/en-us/office/turn-your-presentation-into-a-video-c140551f-cb37-4818-b5d4-3e30815c3e83}{PowerPoint
will do that too}. I can't find a way to do that with LibreOffice,
though. There is mention of an extension to do that in the AOO listings
\href{https://extensions.openoffice.org/en/project/impress-video-converter}{but
it pretty much lists only Windows support}.

Groupware would be used as the theoretical virtual backoffice to make
such a thing work. There would be a need to share calendars, share
photos from events, share audio clips, and collaborate on slides. In
theory the free \href{https://en.wikipedia.org/wiki/ICloud}{iCloud}
service offers all that even if you \emph{don't} have an Apple device.
Nextcloud would be usable but additional services would have to be
layered on top to make things work optimally. Teams would be definitely
usable but would also definitely be pricey.

Yeah, this requires further thought. It would technically be a setback.
At this point I would rather launch \emph{something} rather than
continue with providing nothing at all.
